%\documentclass[onecolumn]{aa} % for a paper on 1 column  
%\documentclass[longauth]{aa} % for the long lists of affiliations
%\documentclass[bibyear]{aa} % if the references are not structured according to the author-year natbib style
%\documentclass{aa}  
%\documentclass{memoir}
%\documentclass[oldfontcommands]{memoir}
%\documentclass[twocolumn]{revtex4-2}
%\documentclass[12pt,a4paper,oldfontcommands]{memoir}
\documentclass[10pt,a4paper]{article}
%\documentclass[10pt,a4paper,onecolumn]{paper}
\usepackage{fullpage}

\usepackage{amsmath}
\usepackage{amssymb}
\usepackage{graphicx}
%\usepackage{txfonts}
\usepackage[colorlinks=true, allcolors=blue]{hyperref}
\usepackage[labelfont=bf]{caption}
\usepackage[noabbrev]{cleveref}
\usepackage{tensor}
\usepackage{derivative}
\usepackage{booktabs}

\usepackage{biblatex}
%\bibliographystyle{plainnat}
\addbibresource{report.bib}

\newcommand\TODO[1]{\textcolor{red}{(\textbf{TODO:} #1)}}
\DeclareMathOperator{\asin}{asin}
\DeclareMathOperator{\sinc}{sinc}
\DeclareMathOperator{\diag}{diag}
%\setlength{\mathindent}{20pt}

\begin{document}

\title{\textbf{Solving the Einstein-Boltzmann equations}\\ \\\normalsize\textit{(AST5220 project report)}}
%\subtitle{Einstein-Boltzmann solver}
\author{Herman Sletmoen}

%\institute{ITA Oslo}
\date{Spring 2023}

\iffalse
\abstract
% context heading (optional)
{Solve Einstein-Boltzmann equations}
% aims heading (mandatory)
{Solve Einstein-Boltzmann equations}
% methods heading (mandatory)
{Solve Einstein-Boltzmann equations}
% results heading (mandatory)
{Solve Einstein-Boltzmann equations}
% conclusions heading (optional)
{Solve Einstein-Boltzmann equations}
\fi

%\keywords{cosmology -- Einstein-Boltzmann equations}

\maketitle
%
%________________________________________________________________

\section{Introduction}

\TODO{WIP for now, eventually weave this together with the remaining milestones}

Einstein field equations:
\begin{equation}
	G\indices{_\mu_\nu} + \Lambda g\indices{_\mu_\nu} = \frac{8 \pi G}{c^4} T\indices{_\mu_\nu}
\label{eq_einstein}
\end{equation}
Energy-momentum tensor:
\begin{equation}
	T\indices{^\mu_\nu} = ...
\label{eq_energy_momentum}
\end{equation}

Except in \cref{sec_supernova}, where we look for constraints on cosmological parameters, we use the Planck's cosmological parameters from 2018 \cite{planckcollaborationPlanck2018Results2020},
given and related by:
\begin{equation}
\begin{aligned}
	& \text{reduced Hubble parameter} & h &= 0.67, \\
	& \text{Hubble parameter} & H_0 &= h \cdot \frac{100\,\mathrm{km}}{\mathrm{s}\,\mathrm{Mpc}} = 67 \frac{\mathrm{km}}{\mathrm{s}\,\mathrm{Mpc}}, \\
	& \text{photon temperature} & T_{\gamma0} &= 2.7255\,K, \\
	& \text{effective neutrino number} & N_\text{eff} &= 3.046, \\
	& \text{matter density parameter} & \Omega_{b0} &= 0.05, \\
	& \text{cold dark matter density parameter} & \Omega_{c0} &= 0.267,\\
	& \text{curvature density parameter} & \Omega_{k0} &= 0, \\
	& \text{photon density parameter} & \Omega_{\gamma0} &= \frac{\pi^2}{15} \cdot \frac{(k_B T_{\gamma0})^4}{\hbar^3 c^5} \cdot \frac{8 \pi G}{3 H_0^2} = 0.000055, \\
	& \text{neutrino density parameter} & \Omega_{\nu0} &= N_{\rm eff}\cdot \frac{7}{8}\left(\frac{4}{11}\right)^{4/3}\Omega_{\gamma 0} = 0.000038, \\
	& \text{cosmological constant density parameter} & \Omega_{\Lambda 0} &= 1 - (\Omega_{k 0}+\Omega_{b 0}+\Omega_{c0}+\Omega_{\gamma 0}+\Omega_{\nu 0}) = 0.683,\\
	& \text{power spectrum spectral index} & n_s &= 0.965, \\
	& \text{power spectrum amplitude} & A_s &= 2.1\cdot 10^{-9}, \\
	& \text{\TODO{}} & Y_p &= 0.245, \\
	& \text{\TODO{}} & z_{\rm reion} &= 8, \\
	& \text{\TODO{}} & \Delta z_{\rm reion} &= 0.5, \\
	& \text{\TODO{}} & z_{\rm He reion} &= 3.5, \\
	& \text{\TODO{}} & \Delta z_{\rm He reion} &= 0.5.
\end{aligned}
\label{eq_planck2018}
\end{equation}


\clearpage

\section{Background cosmology}
\label{sec_background_cosmology}

According to the cosmological principle,
our Universe is spatially homogeneous and isotropic averaged over large distances.
In this section, we will study the cosmology that describes such a universe
with radiation and matter, the cosmological constant and possibly spatial curvature
that are distributed uniformly in space and evolve only in time.

Formally, this is the zeroth-order ``background'' solution of the perturbed, inhomogeneous universe with structure that we aim to describe.

\subsection{Theory}

\subsubsection{Friedmann equation}

The geometry of a \emph{spatially} homogeneous and isotropic universe
is described by the \textbf{Friedmann-Lemaître-Robertson-Walker (FLRW) metric}
\begin{equation}
\begin{split}
	ds^2 &= -c^2 dt^2 + a^2(t) \left[\frac{dr^2}{1 - kr^2} + r^2\left(d\theta^2 + \sin^2\theta \, d\phi^2\right) \right] \\
	     &= a^2(t) \left[-c^2 d\eta^2 + \frac{dr^2}{1 - kr^2} + r^2\left(d\theta^2 + \sin^2\theta \, d\phi^2\right) \right].
\end{split}
\label{eq_flrw}
\end{equation}
It is here written in spherical coordinates $(r,\theta,\phi)$, with curvature $k$, scale factor $a(t)$,
and either cosmic time $t$ or conformal time $\eta$ defined by $d\eta = dt/a$\footnote{I prefer the convention in which conformal time is a time.},
where it is conformal to the Minkowski metric in the flat case $k=0$.

The homogeneous background universe is filled with an ideal fluid that has
density $\rho(t)$, pressure $P(t)$ and thus energy-momentum tensor
\begin{equation}
	T\indices{^\mu_\nu} = \diag\Big[\rho(t),\, -P(t),\, -P(t),\, -P(t)\Big].
\label{eq_energy_momentum_homogeneous}
\end{equation}
The dynamics and evolution of the universe is governed by the Einstein field equations \eqref{eq_einstein},
with the FLRW metric \eqref{eq_flrw} and energy-momentum tensor \eqref{eq_energy_momentum_homogeneous} fed into its left and right sides.
In addition to curvature and the cosmological constant,
we consider a universe with radiation and matter with densities $\rho_r$ and $\rho_m$,
and pressures given by the equations of state $P_m = 0$ and $P_r = \rho_r c^2 / 3$.
In particular, the $00$ and $11$-components give rise to the Friedmann equation for the Hubble parameter
\begin{equation}
	H(t) = \frac{1}{a} \odv{a}{t} = H_0 \sqrt{\Omega_{r0} \, a^{-4} + \Omega_{m0} \, a^{-3} + \Omega_{k0} \, a^{-2} + \Omega_{\Lambda0}},
\label{eq_friedmann}
\end{equation}
where radiation, matter, curvature and the cosmological constant have the (effective) mass densities
\begin{equation}
	\rho_r(t) = \rho_{r0} a^{-4}, \quad
	\rho_m(t) = \rho_{m0} a^{-3}, \quad
	\rho_k(t) = -\frac{3kc^2}{8 \pi G} a^{-2} = \rho_{k0} a^{-2}, \quad
	\rho_\Lambda(t) = \frac{\Lambda c^2}{8 \pi G} = \text{constant},
\end{equation}
and we define the time-dependent density parameters
\begin{equation}
	\Omega_s(t) = \frac{\rho_s(t)}{\rho_\text{crit}(t)}
\label{eq_density_parameters}
\end{equation}
relative to the critical density $\rho_\text{crit}(t) = 3H^2(t)/8\pi G$.
Present-day values are denoted by $F_0=F(t_0)$.

Due to the densities' differing dependencies on the scale factor,
a universe with all four species will be dominated by radiation early on, then matter, then curvature, and finally the cosmological constant.
We are mostly concerned with a flat universe where there is no curvature that can ever dominate,
so we define the scale factors at equality by
\begin{subequations}
\begin{align}
	\rho_r(t) &= \rho_m(t)       \quad \text{at} \quad a = a_{r=m} = \frac{\rho_{r0}}{\rho_{m0}} = \frac{\Omega_{r0}}{\Omega_{m0}}, \\
	\rho_m(t) &= \rho_\Lambda(t) \quad \text{at} \quad a = a_{m=\Lambda} = \left(\frac{\rho_{m0}}{\rho_{\Lambda}}\right)^\frac13 = \left(\frac{\Omega_{m0}}{\Omega_{\Lambda0}}\right)^\frac13.
\end{align}
\label{eq_equality_times}
\end{subequations}

Moreover, the acceleration of the scale factor
\begin{equation}
	\ddot{a} = \odv[2]{a}{t} = -a^2 H_0^2 \Big( 2 \Omega_{r0} a^{-4} + \Omega_{m0} a^{-3} - 2 \Omega_{\Lambda0} \Big).
\label{eq_acceleration}
\end{equation}
will be negative at early times while the expansion slows down,
but when the cosmological constant becomes more important, $\ddot{a} \geq 0$ and the expansion starts to accelerate.

In addition to the ``cosmic'' Hubble parameter \eqref{eq_friedmann},
we define the \textbf{conformal Hubble parameter}
\begin{equation}
	\mathcal{H} = \frac1a \odv{a}{\eta} = \odv{a}{t} = a H.
\label{eq_conformal_hubble}
\end{equation}
For example, in universes dominated by radiation, matter and the cosmological constant,
we have,
\begin{subequations}
\begin{align}
	\mathcal{H} &= H_0 \sqrt{\Omega_{r0}} \, a^{-1} && \Big( \Omega_r \gg \{\Omega_m,\Omega_\Lambda\} \Big), \label{eq_conformal_hubble_dominated_radiation} \\
	\mathcal{H} &= H_0 \sqrt{\Omega_{m0}} \, a^{-1/2} && \Big( \Omega_m \gg \{\Omega_r,\Omega_\Lambda\} \Big), \\
	\mathcal{H} &= H_0 \sqrt{\Omega_{\Lambda0}} \, a && \Big( \Omega_\Lambda \gg \{\Omega_m,\Omega_r\} \Big).
\end{align}
\label{eq_conformal_hubble_dominated}
\end{subequations}
\iffalse
\begin{equation}
	\mathcal{H} = \Bigg\{
		H_0 \sqrt{\Omega_{r0}} a^{-1}, \,\,
		H_0 \sqrt{\Omega_{m0}} a^{-1/2}, \,\,
		H_0 \sqrt{\Omega_{\Lambda0}} a
	\Bigg\}
\label{eq_conformal_hubble_dominated}
\end{equation}
\fi
\iffalse
\begin{equation}
\begin{aligned}
	\mathcal{H} &= H_0 \sqrt{\Omega_{r0}} e^{-x}, &
	\frac{\mathcal{H}'}{\mathcal{H}} &= -1, &
	\frac{\mathcal{H}''}{\mathcal{H}} &= 1 & \Big(\Omega_r \gg \{\Omega_m,\Omega_\Lambda\}\Big) \\
	\mathcal{H} &= H_0 \sqrt{\Omega_{r0}} e^{-\frac12x}, &
	\frac{\mathcal{H}'}{\mathcal{H}} &= -\frac12 &
	\frac{\mathcal{H}''}{\mathcal{H}} &= \frac14 & \Big(\Omega_m \gg \{\Omega_r,\Omega_\Lambda\}\Big) \\
	\mathcal{H} &= H_0 \sqrt{\Omega_{r0}} e^{+x}, &
	\frac{\mathcal{H}'}{\mathcal{H}} &= +1, &
	\frac{\mathcal{H}''}{\mathcal{H}} &= 1 & \Big(\Omega_\Lambda \gg \{\Omega_r,\Omega_m\}\Big).\\
\end{aligned}
\label{eq_conformal_hubble_dominated}
\end{equation}
\fi

\subsubsection{Cosmic and conformal time}

In the form \eqref{eq_friedmann},
the Friedmann equation is a differential equation for the scale factor $a(t)$
as a function of cosmic time $t$.
Instead, we will use its natural logarithm $x = \log a$ (so $a = e^x$) to parametrize the evolution of the universe,
and solve $H = \frac1a \odv{a}{t}$ and $d\eta = dt / a$ for the cosmic and conformal times
\begin{equation}
	t = \int_0^a \frac{da}{aH} = \int_0^x \frac{dx}{H}
	\qquad \text{and} \qquad
	\eta = \int_0^a \frac{da}{a^2 H} = \int_0^x \frac{dx}{aH}.
\label{eq_cosmic_conformal_time}
\end{equation}
In general, these integrals must be computed numerically.
However, in a flat universe with no cosmological constant and radiation-matter equality at $a_\text{eq} = \Omega_{r0}/\Omega_{m0}$,
they can be evaluated analytically to give
\begin{subequations}
\begin{align}
	t &= \int_0^a \frac{da}{H_0 \sqrt{\Omega_{r0}a^{-2} + \Omega_{m0}a^{-1}}}
	   = \frac{2}{3 H_0 \sqrt{\Omega_{m0}}} \Big[\sqrt{a + a_\text{eq}} \big(a - 2 a_\text{eq}\big) + 2 a_\text{eq}^{3/2} \Big]
	\label{eq_cosmic_time_anal}, \\
	\eta &= \int_0^a \frac{da}{H_0 \sqrt{\Omega_{r0} + \Omega_{m0} a}}
		 = \frac{2}{H_0 \sqrt{\Omega_{m0}}} \Big[ \sqrt{a + a_\text{eq}} - \sqrt{a_\text{eq}}\Big]
	\label{eq_conformal_time_anal}.
\end{align}
\label{eq_cosmic_conformal_time_anal}
\end{subequations}
In a universe with only radiation, these expressions reduce further to
\begin{equation}
	t = \frac{a^2}{2 H_0 \sqrt{\Omega_{r0}}}
	\qquad \text{and} \qquad
	\eta = \frac{a}{H_0 \sqrt{\Omega_{r0}}},
\label{eq_cosmic_conformal_time_anal_radiation}
\end{equation}
as can also be found by solving the Friedmann equation \eqref{eq_friedmann} with $\Omega_{m} = \Omega_{k} = \Omega_{\Lambda} = 0$.


\subsection{Implementation}
\label{sec_background_cosmology_implementation}

\begin{itemize}
	\item We represent a $\Lambda$CDM cosmology with an object that takes $h$, $\Omega_{b0}$, $\Omega_{c0}$, $\Omega_{k0}$, $T_{\gamma0}$ and $N_\text{eff}$ as free parameters,
	      and then computes the remaining dependent parameters from the relations \eqref{eq_planck2018}.
	\item Instead of cosmic or conformal time, we use $x = \log a$ as a common internal time variable to parametrize the evolution of the universe.
	      Exchanging $a(t) \leftrightarrow t(a)$ \textbf{requires that $a(t)$ is monotonically increasing}.
	      This \emph{always} holds in a universe with $\Omega_{k0} \geq 0$,
	      because the Hubble parameter \eqref{eq_friedmann} can then never reach zero.
	      However, a universe with sufficiently negative $\Omega_{k0} < 0$ can turn around and begin to collapse when $\dot{a}=0$.
	      Although not relevant for the flat Planck cosmology \eqref{eq_planck2018},
	      we implement a function that checks whether this happens for a set of general cosmological parameters.
	      This comes to use in \cref{sec_supernova}, where we allow for arbitrary curvature.
	\item We compute $\mathcal{H}(x)$ and its derivatives $\odv{\mathcal{H}}/{x}$ and $\odv[2]{\mathcal{H}}/{x}$ analytically based on the Friedmann equation \eqref{eq_friedmann}.
	      The derivatives can be derived without a lot of work by noting that
		  $E(x) = \Omega_{r0} e^{-4x} + \Omega_{m0} e^{-3x} + \Omega_{k0} e^{-2x} + \Omega_{\Lambda0}$
		  in the square root of the Hubble parameter has the derivatives
		  \begin{equation*}
			  \odv[d]{E}{x} = (-4)^d \Omega_{r0} a^{-4}(x) + (-3)^d \Omega_{m0} a^{-3}(x) + (-2)^d \Omega_{k0} a^{-2}(x) \qquad \big(d \geq 1\big).
		  \end{equation*}
	\item We compute the density parameters \eqref{eq_density_parameters} from today's density parameters using, for example for radiation, $\Omega_r = \rho_r / \rho_\text{crit} = (\rho_{r0} a^{-4} / \rho_{\text{crit},0}) (\rho_{\text{crit},0} / \rho_\text{crit}) = \Omega_{r0} a^{-4} (H_0 / H)^2$.
	\item We compute the radiation-matter and matter-cosmological constant equalities \eqref{eq_equality_times} analytically,
	      but the onset of the acceleration \eqref{eq_acceleration} numerically from when $\odv{\mathcal{H}}/{x} = \ddot{a} / H = 0$.
	\item We compute the cosmic time $t(x)$ and conformal time $\eta(x)$ by inserting their derivatives $\odv{t}/{x}$ and $\odv{\eta}/{x}$ into an adaptive 4th(5th)-order Runge-Kutta integrator.
	      As it is computationally infeasible to integrate from $x=-\infty$,
	      we start from a small initial value, like $x = -20$,
	      and the corresponding analytical cosmic or conformal time \eqref{eq_cosmic_conformal_time_anal} in a universe dominated by radiation and matter.
	\item We store the integrated $t(x)$ and $\eta(x)$ on a cubic spline, so subsequent evaluations are fast.
\end{itemize}

\subsection{Tests and results}

We study the evolution of a universe with the Planck 2018 cosmology \eqref{eq_planck2018}.

\begin{table*}
\centering
\caption{%
	The time of occurence of four important events in the evolution of a universe with the Planck cosmology \eqref{eq_planck2018},
	expressed in terms of the scale factor $a$, its natural logarithm $x = \log a$, redshift $z = \frac1a - 1$, cosmic time $t$ and conformal time $\eta$.
}
\label{table_times}
\begin{tabular}{l c c c c c}
	\toprule
	Event                                                               & $x$     & $a$       & $z$    & $\eta / \mathrm{Gyr}$    & $t / \mathrm{Gyr}$ \\
	\midrule
	Radiation-matter equality ($\Omega_r = \Omega_m$)                   & $-8.13$ & $0.0003$  & $3400$ & $0.4$ & $0.00005$ \\
	Acceleration onset ($\ddot{a} = 0$)                                 & $-0.49$ & $0.61$    & $0.63$ & $38.5$ & $7.8$   \\
	Matter-cosmological constant equality ($\Omega_m = \Omega_\Lambda$) & $-0.26$ & $0.77$    & $0.29$ & $42.3$ & $10.4$  \\
	Today ($t = t_0$)                                                   & $0$     & $1$       & $0$    & $46.3$ & $13.9$  \\
	\bottomrule
\end{tabular}
\end{table*}

\begin{figure}
	\centering
	\includegraphics[scale=0.7]{../plots/density_parameters.pdf}
\caption{Evolution of the density parameters \eqref{eq_density_parameters} in the Planck 2018 cosmology \eqref{eq_planck2018}.}
\label{fig_density_parameters}
\end{figure}

\Cref{fig_density_parameters} shows that the universe
transitions from being dominated by radiation to matter to the cosmological constant,
with the equality times \eqref{eq_equality_times} reported in \cref{table_times}.
In this cosmology, there is no curvature $\Omega_{k} = \Omega_{k0} = 0$,
and all density parameters sum to $\Omega_{r} + \Omega_m + \Omega_k + \Omega_\Lambda = 1$ at all times -- as they should, by the Friedmann equation \eqref{eq_friedmann} and definition \eqref{eq_density_parameters}.

\begin{figure}
	\centering
	\includegraphics[scale=0.7]{../plots/conformal_hubble.pdf}
	\includegraphics[scale=0.7]{../plots/conformal_hubble_derivative1.pdf}
	\includegraphics[scale=0.7]{../plots/conformal_hubble_derivative2.pdf}
	\caption{%
		Evolution of the conformal Hubble parameter \eqref{eq_conformal_hubble} and its two derivatives in the Planck 2018 cosmology \eqref{eq_planck2018},
		compared to their values \eqref{eq_conformal_hubble_dominated} in universes dominated by one species.}
	\label{fig_conformal_hubble}
\end{figure}

\Cref{fig_conformal_hubble} shows the evolution of the conformal Hubble parameter \eqref{eq_conformal_hubble} and its two derivatives.
Note that the expansion rate $\dot{a}$ decreases most quickly during radiation domination and slower during matter domination,
but the universe starts to \emph{accelerate} slightly before $\Omega_m = \Omega_\Lambda$, at the time reported in \cref{table_times}.
This is caused by the rise of the cosmological constant, and its effective negative pressure.
Moreover, during the three dominated eras,
the evolution is consistent with the analytical expectation \eqref{eq_conformal_hubble_dominated}.

\begin{figure}[t]
	\centering
	\includegraphics[scale=0.7]{../plots/times.pdf}
	\caption{%
		Evolution of the cosmic and conformal times \eqref{eq_cosmic_conformal_time} in the Planck 2018 cosmology \eqref{eq_planck2018},
		compared to the analytical expressions \eqref{eq_cosmic_conformal_time_anal} in a universe with no cosmological constant.
	}
	\label{fig_cosmic_conformal_time}

	\medskip

	\includegraphics[scale=0.7]{../plots/eta_H.pdf}
	\caption{%
		Evolution of the product between the conformal time \eqref{eq_cosmic_conformal_time} and conformal Hubble parameter \eqref{eq_conformal_hubble},
		compared to that with the analytical time \eqref{eq_conformal_time_anal} and the Hubble parameter with $\Omega_{k0}=\Omega_{\Lambda0}=0$.
	}
	\label{fig_eta_H}
\end{figure}

\Cref{fig_cosmic_conformal_time} shows the relation between the scale factor and cosmic and conformal time \eqref{eq_cosmic_conformal_time} from numerical integration.
Before the cosmological constant becomes important, they closely match the analytical times \eqref{eq_cosmic_conformal_time_anal} from a universe with only radiation and matter.
We can also read off the current age of the universe, as reported in \cref{table_times}.

\Cref{fig_eta_H} shows the evolution of the product $\eta \mathcal{H}$.
The former plots indicate that our computation of conformal time and the Hubble parameter work independently,
and this shows that so does the combination.
Through radiation-domination and matter-domination,
it follows the value we expect from the analytical expression \eqref{eq_conformal_time_anal}
and the Hubble parameter with $\Omega_{k} = \Omega_\Lambda = 0$.
In particular, as $x \rightarrow -\infty$ and radiation dominates,
the product between the conformal time \eqref{eq_cosmic_conformal_time_anal_radiation}
and the conformal Hubble parameter \eqref{eq_conformal_hubble_dominated_radiation} converges to 1.

\clearpage

\section{Cosmological constraints from supernovae}
\label{sec_supernova}

In this section, we forget most of the Planck cosmological parameters \eqref{eq_planck2018},
neglecting neutrinos by fixing $N_\text{eff}=0$ and keeping only $T_{\gamma0}$, hence fixing $\Omega_{r0}$.
Instead, we constrain the independent parameters $h$, $\Omega_{m0}$ and $\Omega_{k0}$,
and hence the dependent $\Omega_{\Lambda 0}=1-\Omega_{k0}-\Omega_{m0}-\Omega_{r0}$,
using observed supernovae luminosity distances from \cite{betouleImprovedCosmologicalConstraints2014}.
To do so, we do a Markov chain Monte Carlo (MCMC) analysis
by stepping through cosmologies with various parameters using the Metropolis-Hastings algorithm
and comparing their predicted luminosity distances to the data.

\subsection{Theory}

\subsubsection{Cosmological distances}

Equipped with the FLRW metric \eqref{eq_flrw} and conformal time \eqref{eq_cosmic_conformal_time},
we can derive how to compute distances in the universe.
Consider a photon traveling on a radial path with $d\theta = d\phi = 0$ from $(\eta,r)$ to us at $(\eta_0, 0)$ along the null geodesic defined by
\begin{equation*}
	0 = ds^2 = a^2(t) \left[ -c^2 d\eta^2 + \frac{dr^2}{1-kr^2} \right].
\end{equation*}
On the comoving grid (in $[\ldots]$), it travels the \textbf{comoving distance}
\begin{equation}
	\chi = \int_{\eta}^{\eta_0} c d\eta = c(\eta_0 - \eta) = \int_r^0 \frac{-dr}{\sqrt{1-kr^2}} = \frac{\arcsin\Big(\sqrt{k}r\Big)}{\sqrt{k}},
\label{eq_comoving_distance}
\end{equation}
so it came from the radial coordinate%
\footnote{This holds for all $k$ as $\sinc(x) = \sin x / x$ takes complex arguments, with $\sin(ix) = i \sinh x$ and $\sinc(0) = 1$.}
\begin{equation}
	r = \frac{\sin\Big(\sqrt{k}\chi\Big)}{\sqrt{k}} = \chi \sinc\Big(\sqrt{k}\chi\Big).
\label{eq_radial_coordinate}
\end{equation}

Given the redshift $z$ of light,
we can then compute the scale factor $a = (z+1)^{-1}$ at emission and its logarithm $x = \log a$,
the corresponding conformal time \eqref{eq_cosmic_conformal_time},
the comoving distance \eqref{eq_comoving_distance}, the radial coordinate \eqref{eq_radial_coordinate}
and finally the corresponding \textbf{angular diameter distance} and \textbf{luminosity distance}
\begin{equation}
	d_A = a r
	\qquad \text{and} \qquad
	d_L = \frac{r}{a} = \frac{d_A}{a^2}.
\label{eq_distances}
\end{equation}

\subsubsection{Statistics}

From \cite{betouleImprovedCosmologicalConstraints2014},
we have measured luminosity distances $d_{L}^\text{obs}(z_i)$ and their
corresponding measurement uncertainties $\sigma_i^\text{obs}$
for $N=31$ different redshifts $z_i$.
Given the three cosmological parameters $h$, $\Omega_{m0}$ and $\Omega_{k0}$,
we can then fit the data to corresponding theoretically predicted distances $d_L(z_i; h, \Omega_{m0}, \Omega_{k0})$.
Assuming the different measurements are Gaussian distributed and uncorrelated,
the likelihood function that rates the fit is $L \propto e^{-\chi^2/2}$, where the $\chi^2$-function is
\begin{equation}
	\chi^2(h,\Omega_{m0},\Omega_{k0}) = \sum_{i=1}^{N} \left( \frac{d_L(z_i; h, \Omega_{m0}, \Omega_{k0}) - d_{L}^\text{obs}(z_i)}{\sigma_i^\text{obs}} \right)^2.
\label{eq_chi2}
\end{equation}

The Metropolis-Hastings algorithm steps through various combinations of $h$, $\Omega_{m0}$ and $\Omega_{k0}$ in parameter space.
Each iteration, it updates the parameters $p_{i+1} = p_i + n_i s_i$,
where $n_i$ is drawn from a normal distribution $N(0,1)$ and $s_i$ are step sizes for each parameter.
With a probability of $\min\big\{L_{i+1}/L_i$, 100\%\big\},
it then records the parameters as a random sample of their probability distributions.
As the probability only depends on $L_{i+1}/L_i$, only the proportionality in $L$ is required!
\TODO{best fit, confidence regions, ...}

\subsection{Implementation}

\begin{itemize}
	\item We run our own simple homemade Metropolis-Hastings algorithm.
	      As input, it takes a function that computes the likelihood's logarithm $\log L(\mathbf{p})$ for a set of parameters $\mathbf{p}$, and their lower and upper bounds.
	      Unless specified explicitly, it guesses step sizes of the parameters as a fixed proportion of their bounds,
		  but adaptively refines them if the algorithm accepts new parameters at a rate that deviates too much from the ``optimal'' acceptance rate around $25\%$ \cite{gelmanWeakConvergenceOptimal1997}.
		  The algorithm can run multiple chains from different initial parameter guesses,
		  each with a requested number of (accepted) samples after removing a given number of burn-in samples.
	\item We exclude parameters outside their specified bounds by assigning $L=0$ to them, so the Metropolis-Hastings algorithm does not consider them.
	\item As mentioned in \cref{sec_background_cosmology_implementation},
	      our implementation of the background cosmology parametrized by the (logarithm of) the scale factor
	      \textbf{cannot handle curved cosmologies with turnaround} $\dot{a} = 0$.
	      These cosmologies can arise now that we allow for $\Omega_{k0} \neq 0$.
	      One example of such a cosmology has $\Omega_{m0} = 0.2$ and $\Omega_{k0} = -0.9$, and thus $\Omega_{\Lambda0} = 1.7$.
	      We check for such cosmologies and exclude them, as well, by assigning $L=0$ to them.
\end{itemize}

\subsection{Results}

\begin{figure}
	\centering
	\includegraphics[scale=0.7]{../plots/supernova_distance.pdf}
	\caption{Observed and predicted luminosity distances \eqref{eq_distances} from \cite{betouleImprovedCosmologicalConstraints2014} and the Planck cosmology \eqref{eq_planck2018}.}
	\label{fig_luminosity_distances}
\end{figure}

\Cref{fig_luminosity_distances} shows observed and predicted luminosity distances from the Planck 2018 cosmology \eqref{eq_planck2018}.
The agreement is relatively good, so we expect that our constraints should be close to their values.

\begin{figure}
	\centering
	\includegraphics[scale=0.7]{../plots/supernova_hubble.pdf}
	\includegraphics[scale=0.7]{../plots/supernova_omegas.pdf}
	\caption{%
		Posterior probability distribution of today's reduced Hubble parameter $h$,
		and constraints on $\Omega_{m0}$ versus $\Omega_{\Lambda0}$,
		from $10 \times 10000$ samples of the Metropolis-Hastings algorithm with $\log L = -\chi^2/2$ and the $\chi^2$ sum \eqref{eq_chi2}.
	}
	\label{fig_supernova_mcmc}
\end{figure}

\Cref{fig_supernova_mcmc} shows our MCMC constraints on $h$, $\Omega_{m0}$ and $\Omega_\Lambda$.
Note that the constraint in the $\Omega_{m0}$-$\Omega_{\Lambda0}$-plane is highly orthogonal to the line of flat universes,
so supernova data can give good constraints when combined with some other constraint that argues in favor of flatness, for example.
Our best fits for $\Omega_{m0}$ and $\Omega_{\Lambda0}$ agrees relatively well with both Planck's and a similar analysis in \cite[Fig. 15]{betouleImprovedCosmologicalConstraints2014}.
On the other hand, our Hubble parameter is somewhat larger than Planck's and exemplifies the Hubble tension.

\section{Conclusions}

\TODO{}

%\bibliography{report.bib}
\printbibliography

\end{document}
