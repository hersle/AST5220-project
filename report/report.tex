%\documentclass[onecolumn]{aa} % for a paper on 1 column  
%\documentclass[longauth]{aa} % for the long lists of affiliations
%\documentclass[bibyear]{aa} % if the references are not structured according to the author-year natbib style
%\documentclass{aa}  
%\documentclass{memoir}
%\documentclass[oldfontcommands]{memoir}
%\documentclass[twocolumn]{revtex4-2}
%\documentclass[12pt,a4paper,oldfontcommands]{memoir}
\documentclass[10pt,a4paper]{article}
%\documentclass[10pt,a4paper,onecolumn]{paper}
\usepackage{fullpage}

\usepackage{amsmath}
\usepackage{amssymb}
\usepackage{graphicx}
%\usepackage{txfonts}
\usepackage[colorlinks=true, allcolors=blue]{hyperref}
\usepackage[labelfont=bf]{caption}
\usepackage[noabbrev]{cleveref}
\usepackage{tensor}
\usepackage{derivative}
\usepackage{booktabs}

\usepackage{biblatex}
%\bibliographystyle{plainnat}
\addbibresource{report.bib}

\newcommand\TODO[1]{\textcolor{red}{(\textbf{TODO:} #1)}}
\DeclareMathOperator{\asin}{asin}
\DeclareMathOperator{\sinc}{sinc}
\DeclareMathOperator{\diag}{diag}
%\setlength{\mathindent}{20pt}

\begin{document}

\title{AST5220 project report}
%\subtitle{Einstein-Boltzmann solver}
\author{Herman Sletmoen}

%\institute{ITA Oslo}
\date{Spring 2023}

\iffalse
\abstract
% context heading (optional)
{Solve Einstein-Boltzmann equations}
% aims heading (mandatory)
{Solve Einstein-Boltzmann equations}
% methods heading (mandatory)
{Solve Einstein-Boltzmann equations}
% results heading (mandatory)
{Solve Einstein-Boltzmann equations}
% conclusions heading (optional)
{Solve Einstein-Boltzmann equations}
\fi

%\keywords{cosmology -- Einstein-Boltzmann equations}

\maketitle
%
%________________________________________________________________

\section{Introduction}

\TODO{Eventually weave this together with the rest}

Einstein field equations:
\begin{equation}
	G\indices{_\mu_\nu} + \Lambda g\indices{_\mu_\nu} = \frac{8 \pi G}{c^4} T\indices{_\mu_\nu}
\label{eq_einstein}
\end{equation}
Energy-momentum tensor:
\begin{equation}
	T\indices{^\mu_\nu} = ...
\label{eq_energy_momentum}
\end{equation}

Except in \cref{sec_supernova}, where we look for constraints on cosmological parameters, we use the Planck's cosmological parameters from 2018 \cite{planckcollaborationPlanck2018Results2020},
given and related by:
\begin{equation}
\begin{aligned}
	& \text{reduced Hubble parameter} & h &= 0.67, \\
	& \text{Hubble parameter} & H_0 &= h \cdot \frac{100\,\mathrm{km}}{\mathrm{s}\,\mathrm{Mpc}} = 67 \frac{\mathrm{km}}{\mathrm{s}\,\mathrm{Mpc}}, \\
	& \text{photon temperature} & T_{\gamma0} &= 2.7255\,K, \\
	& \text{neutrino \TODO{?}} & N_\text{eff} &= 3.046, \\
	& \text{matter density parameter} & \Omega_{b0} &= 0.05, \\
	& \text{cold dark matter density parameter} & \Omega_{c0} &= 0.267,\\
	& \text{curvature density parameter} & \Omega_{k0} &= 0, \\
	& \text{photon density parameter} & \Omega_{\gamma0} &= \frac{\pi^2}{15} \cdot \frac{(k_B T_{\gamma0})^4}{\hbar^3 c^5} \cdot \frac{8 \pi G}{3 H_0^2} = 0.000055, \\
	& \text{neutrino density parameter} & \Omega_{\nu0} &= N_{\rm eff}\cdot \frac{7}{8}\left(\frac{4}{11}\right)^{4/3}\Omega_{\gamma 0} = 0.000038, \\
	& \text{cosmological constant density parameter} & \Omega_{\Lambda 0} &= 1 - (\Omega_{k 0}+\Omega_{b 0}+\Omega_{c0}+\Omega_{\gamma 0}+\Omega_{\nu 0}) = 0.683,\\
	& \text{power spectrum spectral index} & n_s &= 0.965, \\
	& \text{power spectrum amplitude} & A_s &= 2.1\cdot 10^{-9}, \\
	& \text{\TODO{}} & Y_p &= 0.245, \\
	& \text{\TODO{}} & z_{\rm reion} &= 8, \\
	& \text{\TODO{}} & \Delta z_{\rm reion} &= 0.5, \\
	& \text{\TODO{}} & z_{\rm He reion} &= 3.5, \\
	& \text{\TODO{}} & \Delta z_{\rm He reion} &= 0.5.
\end{aligned}
\label{eq_planck2018}
\end{equation}


\section{Background cosmology}
\label{sec_background_cosmology}

According to the cosmological principle,
our Universe is spatially homogeneous and isotropic averaged over large distances.
In this section, we will study the cosmology that describes such a universe
with radiation and matter, the cosmological constant and possibly spatial curvature
that are distributed uniformly in space and evolve only in time.

Formally, this is the zeroth-order ``background'' solution of the perturbed, inhomogeneous universe with structure that we aim to describe.
\TODO{?} 

\subsection{Theory}

\subsubsection{Friedmann equation}

The geometry of a \emph{spatially} homogeneous and isotropic universe
is described by the \textbf{Friedmann-Lemaître-Robertson-Walker (FLRW) metric}
\begin{equation}
\begin{split}
	ds^2 &= -c^2 dt^2 + a^2(t) \left[\frac{dr^2}{1 - kr^2} + r^2\left(d\theta^2 + \sin^2\theta \, d\phi^2\right) \right] \\
	     &= a^2(t) \left[-c^2 d\eta^2 + \frac{dr^2}{1 - kr^2} + r^2\left(d\theta^2 + \sin^2\theta \, d\phi^2\right) \right].
\end{split}
\label{eq_flrw}
\end{equation}
It is here written in spherical coordinates $(r,\theta,\phi)$, with curvature $k$, scale factor $a(t)$,
and either cosmic time $t$ or conformal time $\eta$ defined by $d\eta = dt/a$\footnote{I prefer the convention in which conformal time is a time.},
where it is conformal to the Minkowski metric in the flat case $k=0$.

The homogeneous background universe is filled with an ideal fluid that has
energy density $\rho(t)$, pressure $P(t)$ and thus energy-momentum tensor
\begin{equation}
	T\indices{^\mu_\nu} = \diag\Big[\rho(t),\, -P(t),\, -P(t),\, -P(t)\Big].
\label{eq_energy_momentum_homogeneous}
\end{equation}
The dynamics and evolution of the universe is governed by the Einstein field equations \eqref{eq_einstein}.
In particular, its $00$-component gives rise to the Friedmann equation for the Hubble parameter
\begin{equation}
	H(t) = \frac{1}{a} \odv{a}{t} = H_0 \sqrt{\Omega_{r0} \, a^{-4} + \Omega_{m0} \, a^{-3} + \Omega_{k0} \, a^{-2} + \Omega_{\Lambda0}},
\label{eq_friedmann}
\end{equation}
\TODO{accel, derive $a$-dep of $\rho$}
where we define the time-dependent density parameters
\begin{equation}
	\Omega_s(t) = \frac{\rho_s(t)}{\rho_\text{crit}(t)}
\label{eq_density_parameters}
\end{equation}
for radiation, matter, curvature and the cosmological constant
with (effective) energy densities
\begin{equation}
	\rho_r(t) = \rho_{r0} a^{-4}, \quad
	\rho_k(t) = -\frac{3kc^2}{8 \pi G} a^{-2} = \rho_{k0} a^{-2}, \quad
	\rho_m(t) = \rho_{m0} a^{-3}, \quad
	\rho_\Lambda(t) = \frac{\Lambda c^2}{8 \pi G} = \text{constant},
\end{equation}
relative to the critical density
$\rho_\text{crit}(t) = 3H^2(t)/8\pi G$.
Present-day values are denoted by $F_0=F(t_0)$.

Due to the densities' differing dependencies on the scale factor,
a universe with all four species will be dominated by radiation early on, then matter, then curvature, and finally the cosmological constant.
We are mostly concerned with a flat universe where there is no curvature that can ever dominate,
so we define the scale factors at equality by
\begin{subequations}
\begin{align}
	\rho_r(t) &= \rho_m(t)       \quad \text{at} \quad a = a_{r=m} = \frac{\rho_{r0}}{\rho_{m0}} = \frac{\Omega_{r0}}{\Omega_{m0}}, \\
	\rho_m(t) &= \rho_\Lambda(t) \quad \text{at} \quad a = a_{m=\Lambda} = \left(\frac{\rho_{m0}}{\rho_{\Lambda}}\right)^\frac13 = \left(\frac{\Omega_{m0}}{\Omega_{\Lambda0}}\right)^\frac13.
\end{align}
\label{eq_equality_times}
\end{subequations}
\TODO{acceleration}




In addition to the ``cosmic'' Hubble parameter \eqref{eq_friedmann},
we define the \textbf{conformal Hubble parameter}
\begin{equation}
	\mathcal{H} = \frac1a \odv{a}{\eta} = \odv{a}{t} = a H.
\label{eq_conformal_hubble}
\end{equation}
For example, in universes dominated by radiation, matter and the cosmological constant,
we have,
\begin{subequations}
\begin{align}
	\mathcal{H} &= H_0 \sqrt{\Omega_{r0}} \, a^{-1} && \Big( \Omega_r \gg \{\Omega_m,\Omega_\Lambda\} \Big), \label{eq_conformal_hubble_dominated_radiation} \\
	\mathcal{H} &= H_0 \sqrt{\Omega_{m0}} \, a^{-1/2} && \Big( \Omega_m \gg \{\Omega_r,\Omega_\Lambda\} \Big), \\
	\mathcal{H} &= H_0 \sqrt{\Omega_{\Lambda0}} \, a && \Big( \Omega_\Lambda \gg \{\Omega_m,\Omega_r\} \Big).
\end{align}
\label{eq_conformal_hubble_dominated}
\end{subequations}
\iffalse
\begin{equation}
	\mathcal{H} = \Bigg\{
		H_0 \sqrt{\Omega_{r0}} a^{-1}, \,\,
		H_0 \sqrt{\Omega_{m0}} a^{-1/2}, \,\,
		H_0 \sqrt{\Omega_{\Lambda0}} a
	\Bigg\}
\label{eq_conformal_hubble_dominated}
\end{equation}
\fi
\iffalse
\begin{equation}
\begin{aligned}
	\mathcal{H} &= H_0 \sqrt{\Omega_{r0}} e^{-x}, &
	\frac{\mathcal{H}'}{\mathcal{H}} &= -1, &
	\frac{\mathcal{H}''}{\mathcal{H}} &= 1 & \Big(\Omega_r \gg \{\Omega_m,\Omega_\Lambda\}\Big) \\
	\mathcal{H} &= H_0 \sqrt{\Omega_{r0}} e^{-\frac12x}, &
	\frac{\mathcal{H}'}{\mathcal{H}} &= -\frac12 &
	\frac{\mathcal{H}''}{\mathcal{H}} &= \frac14 & \Big(\Omega_m \gg \{\Omega_r,\Omega_\Lambda\}\Big) \\
	\mathcal{H} &= H_0 \sqrt{\Omega_{r0}} e^{+x}, &
	\frac{\mathcal{H}'}{\mathcal{H}} &= +1, &
	\frac{\mathcal{H}''}{\mathcal{H}} &= 1 & \Big(\Omega_\Lambda \gg \{\Omega_r,\Omega_m\}\Big).\\
\end{aligned}
\label{eq_conformal_hubble_dominated}
\end{equation}
\fi

\subsubsection{Cosmic and conformal time}

In the form \eqref{eq_friedmann},
the Friedmann equation is a differential equation for the scale factor $a(t)$
as a function of cosmic time $t$.
Instead, we will use its natural logarithm $x = \log a$ (so $a = e^x$) to parametrize the evolution of the universe,
and solve $H = \frac1a \odv{a}{t}$ and $d\eta = dt / a$ for the cosmic and conformal times
\begin{equation}
	t = \int_0^a \frac{da}{aH} = \int_0^x \frac{dx}{H}
	\qquad \text{and} \qquad
	\eta = \int_0^a \frac{da}{a^2 H} = \int_0^x \frac{dx}{aH}.
\label{eq_cosmic_conformal_time}
\end{equation}
In general, these integrals must be computed numerically.
However, in a flat universe with no cosmological constant and radiation-matter equality at $a_\text{eq} = \Omega_{r0}/\Omega_{m0}$,
they can be evaluated analytically to give
\begin{subequations}
\begin{align}
	t &= \int_0^a \frac{da}{H_0 \sqrt{\Omega_{r0}a^{-2} + \Omega_{m0}a^{-1}}}
	   = \frac{2}{3 H_0 \sqrt{\Omega_{m0}}} \Big[\sqrt{a + a_\text{eq}} \big(a - 2 a_\text{eq}\big) + 2 a_\text{eq}^{3/2} \Big]
	\label{eq_cosmic_time_anal}, \\
	\eta &= \int_0^a \frac{da}{H_0 \sqrt{\Omega_{r0} + \Omega_{m0} a}}
		 = \frac{2}{H_0 \sqrt{\Omega_{m0}}} \Big[ \sqrt{a + a_\text{eq}} - \sqrt{a_\text{eq}}\Big]
	\label{eq_conformal_time_anal}.
\end{align}
\label{eq_cosmic_conformal_time_anal}
\end{subequations}
In a universe with only radiation, these expressions reduce further to
\begin{equation}
	t = \frac{a^2}{2 H_0 \sqrt{\Omega_{r0}}}
	\qquad \text{and} \qquad
	\eta = \frac{a}{H_0 \sqrt{\Omega_{r0}}},
\label{eq_cosmic_conformal_time_anal_radiation}
\end{equation}
as can also be found by solving the Friedmann equation \eqref{eq_friedmann} with $\Omega_{m} = \Omega_{k} = \Omega_{\Lambda} = 0$.


\subsection{Implementation}
\label{sec_background_cosmology_implementation}

\begin{itemize}
	\item We use $x = \log a$ as a common internal time variable that most functions accept.
	\item We represent a $\Lambda$CDM cosmology with an object that takes $h$, $\Omega_{b0}$, $\Omega_{c0}$, $\Omega_{k0}$, $T_{\gamma0}$ and $N_\text{eff}$ as free parameters, and then computes the remaining dependent parameters \eqref{eq_planck2018}.
	\item We compute $\mathcal{H}(x)$ and its derivatives $\odv{\mathcal{H}}/{x}$ and $\odv[2]{\mathcal{H}}/{x}$ analytically based on the Friedmann equation \eqref{eq_friedmann}.
	      The derivatives are easy to write down with the observation that
		  $E(x) = \Omega_{r0} e^{-4x} + \Omega_{m0} e^{-3x} + \Omega_{k0} e^{-2x} + \Omega_{\Lambda0}$
		  in the Hubble parameter has the derivatives
		  \begin{equation*}
			  \odv[d]{E}{x} = (-4)^d \Omega_{r0} a^{-4}(x) + (-3)^d \Omega_{m0} a^{-3}(x) + (-2)^d \Omega_{k0} a^{-2}(x) \qquad \big(d \geq 1\big).
		  \end{equation*}
	\item We compute the density parameters \eqref{eq_density_parameters} from today's density parameters using, for example, $\Omega_r = \rho_r / \rho_\text{crit} = (\rho_{r0} a^{-4} / \rho_{\text{crit},0}) (\rho_{\text{crit},0} / \rho_\text{crit}) = \Omega_{r0} a^{-4} (H_0 / H)^2$.
	\item We compute the cosmic time $t(x)$ and conformal time $\eta(x)$ by inserting their derivatives $\odv{t}/{x}$ and $\odv{\eta}/{x}$ into an adaptive 4th(5th)-order Runge-Kutta integrator.
	      As it is computationally infeasible to start at $x=-\infty$, we pick a small initial value for $x$, like $x = -20$,
	      and compute its corresponding cosmic time \eqref{eq_cosmic_time_anal} or conformal time \eqref{eq_conformal_time_anal} in a radiation and matter-dominated universe.
	      We use this value to start off the numerical integrator.
	\item We compute the radiation-matter and matter-cosmological constant equalities analytically \TODO{},
	      but the acceleration onset numerically.
	\item We exclude crazy cosmologies where the contents of the square root in the Friedmann equation \eqref{eq_friedmann} become negative for some $a > 0$, and the scale factor develops an imaginary part.
	      This unphysical behavior can happen in curved cosmologies, for example with $\Omega_{r0} = 0$, $\Omega_{m0} = 0.2$, $\Omega_{k0}=-0.9$ and $\Omega_{\Lambda} = 1.7$.
		  Although we are mostly concerned with the flat Planck cosmology \eqref{eq_planck2018}, this will come in handy when studying arbitrary cosmologies in \cref{sec_supernova}.
	\item We spline the integrated $t(x)$ and $\eta(x)$ on a cubic spline, so subsequent computations are both fast and accurate.
\end{itemize}

\subsection{Tests and results}

We now construct a background cosmology with the Planck 2018 parameters \eqref{eq_planck2018}
and study the evolution of various quantities in this universe as a function of $x = \log a$.

\begin{table*}
\centering
\caption{%
	The time of occurence of four important events in the evolution of a universe with the Planck cosmology \eqref{eq_planck2018},
	expressed in terms of the scale factor $a$, its natural logarithm $x = \log a$, redshift $z = \frac1a - 1$, cosmic time $t$ and conformal time $\eta$.
}
\label{table_times}
\begin{tabular}{l c c c c c}
	\toprule
	Event                                                               & $x$     & $a$       & $z$    & $\eta / \mathrm{Gyr}$    & $t / \mathrm{Gyr}$ \\
	\midrule
	Radiation-matter equality ($\Omega_r = \Omega_m$)                   & $-8.13$ & $0.0003$  & $3400$ & $0.4$ & $0.00005$ \\
	Acceleration onset ($\ddot{a} = 0$)                                 & $-0.49$ & $0.61$    & $0.63$ & $38.5$ & $7.8$   \\
	Matter-cosmological constant equality ($\Omega_m = \Omega_\Lambda$) & $-0.26$ & $0.77$    & $0.29$ & $42.3$ & $10.4$  \\
	Today ($t = t_0$)                                                   & $0$     & $1$       & $0$    & $46.3$ & $13.9$  \\
	\bottomrule
\end{tabular}
\end{table*}

\begin{figure}
	\centering
	\includegraphics[scale=0.7]{../plots/density_parameters.pdf}
\caption{Evolution of the density parameters \eqref{eq_density_parameters} in the Planck 2018 cosmology \eqref{eq_planck2018}.}
\label{fig_density_parameters}
\end{figure}

\Cref{fig_density_parameters} shows that the universe
transitions from being dominated by radiation to matter to the cosmological constant,
with the equality times \eqref{eq_equality_times} reported in \cref{table_times}.
In this cosmology, there is no curvature $\Omega_{k} = \Omega_{k0} = 0$,
and all density parameters sum to $\Omega_{r} + \Omega_m + \Omega_k + \Omega_\Lambda = 1$ at all times -- as they should, by the Friedmann equation \eqref{eq_friedmann} and definition \eqref{eq_density_parameters}.

\begin{figure}
	\centering
	\includegraphics[scale=0.7]{../plots/conformal_hubble.pdf}
	\includegraphics[scale=0.7]{../plots/conformal_hubble_derivative1.pdf}
	\includegraphics[scale=0.7]{../plots/conformal_hubble_derivative2.pdf}
	\caption{%
		Evolution of the conformal Hubble parameter \eqref{eq_conformal_hubble} and its two derivatives in the Planck 2018 cosmology \eqref{eq_planck2018},
		compared to their values \eqref{eq_conformal_hubble_dominated} in universes dominated by one species.}
	\label{fig_conformal_hubble}
\end{figure}

\Cref{fig_conformal_hubble} shows the evolution of the conformal Hubble parameter and its two derivatives.
Note how the expansion rate $\dot{a}$ decreases most quickly during radiation domination,
slower during matter domination,
and then begins to \emph{increase} slightly before $\Omega_m = \Omega_\Lambda$ and throughout cosmological constant-domination.
This means that the universe begins to \emph{accelerate} when $\ddot{a}=0$, or $\odv{\mathcal{H}}/{x}$, at the time reported in \cref{table_times}.
This is caused by the negative pressure of the cosmological constant.
Moreover, the evolution is consistent with the values \eqref{eq_conformal_hubble_dominated} in universes dominated by one species.

\begin{figure}[t]
	\centering
	\includegraphics[scale=0.7]{../plots/times.pdf}
	\caption{Evolution of the cosmic and conformal times \eqref{eq_cosmic_conformal_time} in the Planck 2018 cosmology \eqref{eq_planck2018}, compared to the analytical expressions \eqref{eq_cosmic_time_anal} in a universe with no cosmological constant.}
	\label{fig_cosmic_conformal_time}

	\includegraphics[scale=0.7]{../plots/eta_H.pdf}
	\caption{%
		Evolution of the product between the conformal time \eqref{eq_cosmic_conformal_time} and conformal Hubble parameter \eqref{eq_conformal_hubble},
		compared to that with the analytical time \eqref{eq_conformal_time_anal} and the Hubble parameter with $\Omega_{k0}=\Omega_{\Lambda0}=0$.
	}
	\label{fig_eta_H}
\end{figure}

\Cref{fig_cosmic_conformal_time} shows the relation between the scale factor and cosmic and conformal time \eqref{eq_cosmic_conformal_time} from numerical integration.
Before the cosmological constant becomes important, they closely match the analytical times \eqref{eq_cosmic_conformal_time_anal} from a universe with only radiation and matter.
We can also read off the current age of the universe, as reported in \cref{table_times}.

\Cref{fig_eta_H} shows the evolution of the combined quantity $\eta \mathcal{H}$.
The former plots indicate that our computation of conformal time and the Hubble parameter work independently,
and this plot shows that so does their product.
Through radiation-domination and matter-domination, it follows the expected value from the analytical expression \eqref{eq_conformal_time_anal}.
In particular, as far back in time as possible, the product between the radiation-dominated conformal time \eqref{eq_cosmic_conformal_time_anal_radiation}
and \eqref{eq_conformal_hubble_dominated_radiation} converges to 1.

\clearpage

\section{Cosmological constraints from supernovae}
\label{sec_supernova}

In this section, we forget most of the Planck 2018 cosmological parameters \eqref{eq_planck2018},
keeping only $T_{\gamma0}=2.7255\,\mathrm{K}$ and hence $\Omega_{\gamma0}$,
and neglecting neutrinos by fixing $N_\text{eff}=0$.
Instead, we constrain the independent parameters $h$, $\Omega_{m0}=\Omega_{b0}+\Omega_{c0}$ and $\Omega_{k0}$,
and hence the dependent $\Omega_{\Lambda 0}=1-\Omega_{k0}-\Omega_{m0}-\Omega_{r0}$,
using observed supernovae luminosity distances from \cite{betouleImprovedCosmologicalConstraints2014}.
To do so, we step through cosmologies with various parameters using the Metropolis-Hastings algorithm,
and compare their predicted luminosity distances to the data.

\subsection{Theory}

\subsubsection{Cosmological distances}

Equipped with the FLRW metric \eqref{eq_flrw} and conformal time \eqref{eq_cosmic_conformal_time},
we can derive how to compute distances in the universe.
Consider a photon traveling on a radial path with $d\theta = d\phi = 0$ from $(\eta,r)$ to us at $(\eta_0, 0)$ along the null geodesic defined by
\begin{equation*}
	0 = ds^2 = a^2(t) \left[ -c^2 d\eta^2 + \frac{dr^2}{1-kr^2} \right].
\end{equation*}
On the comoving grid (in $[\ldots]$), it travels the \textbf{comoving distance}
\begin{equation}
	\chi = \int_{\eta}^{\eta_0} c d\eta = c(\eta_0 - \eta) = \int_r^0 \frac{-dr}{\sqrt{1-kr^2}} = \frac{\arcsin\Big(\sqrt{k}r\Big)}{\sqrt{k}},
\label{eq_comoving_distance}
\end{equation}
so it came from the radial coordinate
\begin{equation}
	r = \frac{\sin\Big(\sqrt{k}\chi\Big)}{\sqrt{k}} = \chi \sinc\Big(\sqrt{k}\chi\Big).
\label{eq_radial_coordinate}
\end{equation}
This equation holds for $k$ of \emph{all} signs with the definition $\sinc(x) = \sin x / x$,
its limit $\sinc(x=0)=1$ and the complex substitution $\sin(ix) = i \sinh x$.

Given the redshift $z$ of light,
we can then compute the scale factor $a = (z+1)^{-1}$ at emission and its logarithm $x = \log a$,
the corresponding conformal time \eqref{eq_cosmic_conformal_time},
the comoving distance \eqref{eq_comoving_distance}, the radial coordinate \eqref{eq_radial_coordinate}
and finally the corresponding angular diameter distance and luminosity distance
\begin{equation}
	d_A = a r
	\qquad \text{and} \qquad
	d_L = \frac{r}{a} = \frac{d_A}{a^2}.
\label{eq_distances}
\end{equation}

\subsubsection{Statistics}

From \cite{betouleImprovedCosmologicalConstraints2014},
we have measured luminosity distances $d_{L}^\text{obs}(z_i)$ and their
corresponding measurement uncertainties $\sigma_i^\text{obs}$
for $N=31$ different redshifts $z_i$.
Given the three cosmological parameters $h$, $\Omega_{m0}$ and $\Omega_{k0}$,
we can then predict corresponding theoretical distances $d_L(z_i; h, \Omega_{m0}, \Omega_{k0})$.
Assuming the different measurements are Gaussian distributed and uncorrelated,
the likelihood function that quantifies how good the data fits the theory has likelihood $L \propto e^{-\chi^2/2}$, where the $\chi^2$-function is
\begin{equation}
	\chi^2(h,\Omega_{m0},\Omega_{k0}) = \sum_{i=1}^{N} \left( \frac{d_L(z_i; h, \Omega_{m0}, \Omega_{k0}) - d_{L}^\text{obs}(z_i)}{\sigma_i^\text{obs}} \right)^2.
\label{eq_chi2}
\end{equation}

The Metropolis-Hastings algorithm steps through various combinations of $h$, $\Omega_{m0}$ and $\Omega_{k0}$ in parameter space.
Each iteration, it updates the parameters $p_{i+1} = p_i + n_i s_i$,
where $n_i$ is drawn from a normal distribution $N(0,1)$ and $s_i$ are step sizes for each parameter.
With a probability of $\min\big\{L_{i+1}/L_i$, 100\%\big\},
it then records the parameters as a random sample of their probability distributions.
As the probability only depends on $L_{i+1}/L_i$, only the proportionality in $L$ is required!

\subsection{Implementation}

\begin{itemize}
	\item We implement a simple and generic homemade version of the Metropolis-Hastings algorithm.
	      As input, it takes a function that computes the log-likelihood $\log L(\mathbf{p})$ for a set of parameters $\mathbf{p}$,
	      and lower and upper bounds on the parameters.
	      It guesses step sizes of the parameters as a fixed proportion of their bounds,
		  and then adaptively refines them if the algorithm accepts new parameters that deviates too much from the ``optimal'' acceptance rate around $25\%$ \cite{gelmanWeakConvergenceOptimal1997}.
		  The algorithm runs one or multiple chains with a requested number of (accepted) samples from initial parameter guesses,
		  and removes a requested number of burn-in samples.
	\item When we run into ``crazy cosmologies'' (as described in \cref{sec_background_cosmology_implementation}) or cosmologies whose parameters are outside the allowed bounds (for example cosmologies with $\Omega_{m0} < 0$), we assign zero likelihood $L=0$, so they are not accepted.
\end{itemize}

\subsection{Results}

\begin{figure}
	\centering
	\includegraphics[scale=0.7]{../plots/supernova_distance.pdf}
	\caption{Observed luminosity distances from supernovae from \cite{betouleImprovedCosmologicalConstraints2014}, compared to predictions \eqref{eq_distances} from the Planck 2018 cosmology \eqref{eq_planck2018}.}
	\label{fig_luminosity_distances}
\end{figure}

Before turning to constraining the cosmological parameters,
we compare the observed distances to those predicted with the Planck 2018 cosmology \eqref{eq_planck2018} in \cref{fig_luminosity_distances}.
The agreement is already relatively good, so we expect that our constraints and best fit should be close to their values.

\begin{figure}
	\centering
	\includegraphics[scale=0.7]{../plots/supernova_hubble.pdf}
	\includegraphics[scale=0.7]{../plots/supernova_omegas.pdf}
	\caption{%
		Posterior probability distribution of today's reduced Hubble parameter $h$,
		and constraints on $\Omega_{m0}$ versus $\Omega_{\Lambda0}$,
		from $5 \times 10000$ samples of the Metropolis-Hastings algorithm with $\log L = -\chi^2/2$ and the chi-squared sum \eqref{eq_chi2}.
	}
	\label{fig_supernova_mcmc}
\end{figure}

We now run the Metropolis-Hastings algorithm with five chains of 10000 samples each, neglecting $1000$ initial samples as burn-in.
This gives the probability distributions for $h$ and confidence regions for $\Omega_{m0}$ and $\Omega_{\Lambda0}$ in \cref{fig_supernova_mcmc}.
Note that the constraint in the $\Omega_{m0}$-$\Omega_{\Lambda0}$-plane is highly orthogonal to the line of flat universes,
so supernova data can give good constraints when combined with some other constraint that argues in favor of flatness, for example.
Our best fits for $\Omega_{m0}$ and $\Omega_{\Lambda0}$ agrees quite well with Planck's,
while our Hubble parameter is somewhat larger and exemplifies the Hubble tension.
\TODO{count $\sigma$?}

\section{Conclusions}

%\bibliography{report.bib}
\printbibliography

\end{document}
